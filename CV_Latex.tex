\documentclass[11pt,a4paper]{moderncv}
\moderncvtheme[blue]{classic}              
\usepackage[utf8]{inputenc}
\usepackage{moderntimeline,tikz}

\usepackage[top=1.1cm, bottom=1.1cm, left=1.5cm, right=1.5cm]{geometry}

\setlength{\hintscolumnwidth}{3cm}
\photo[64pt]{Photo.png}

\definecolor{linecolor}{RGB}{128,128,128}
\tlmaxdates{2006}{2015}
%\tlwidth{0.4ex}
%\tltext{\tiny}

\nopagenumbers{}

\firstname{Ludovic}
\familyname{Lardiès}
\title{À la recherche d'opportunités pour la rentrée de septembre 2015}              
\address{20 rue Octave David}{25000 Besançon}    
\email{ludovic@lardies.fr}                    
\mobile{(033) 06 88 65 64 32} 
\homepage{http://lardies.fr}
\extrainfo{Né le 17/10/1990 \\ Permis B}


\begin{document}

\maketitle
\vspace{-3.5em}


\section{Formation}

\tllabelcventry[linecolor]{2011.9}{2014}{2012-2015}{Ingénieur en Informatique}{Université de Technologie de Belfort-Montbéliard, UTBM (90)}{}{}{}

\tllabelcventry[linecolor]{2010}{2012}{2010-2012}{BTS Informatique et Réseaux pour l'Industrie et les Services techniques (IRIS)}{Lycée Jules Haag, Besançon (25)}{}{}{}

\tllabelcventry[linecolor]{2008}{2010}{2008-2010}{Baccalauréat (Mention AB)}{Lycée Jules Haag, Besançon}{}{}{Série STI, Filière Microtechniques} 

\section{Expérience professionnelle}

\cventry{Printemps 2015\\ (6 mois)}{Projet de fin d'études -- cursus UTBM}{}{Tech4Team, Paris} {}{Parsing et importation de données clients (fichiers CSV ou utilisation d'API) dans une base de données (en Python et Ruby). Scrapping de sites en Python. Conception et amélioration de bases de données}

\cventry{Automne 2014}{Projet UTBM}{}{projet de développement réalisé pour le service de chirurgie cardiaque du CHU de Strasbourg} {}{Application Android permetant d'effectuer un échange de questions/réponses avec l'utilisateur \texttt{https://github.com/ludovicl/TO52-Android-Conversation.git}}

\cventry{Été 2014\\ (2 mois)}{Intérim : développeur informatique}{}{Parkeon, Besançon} {}{Création de la documentation d'un SDK. Participation au changement de compilateur et de chaîne de compilation (GNU et CMake)}

\cventry{Automne 2013\\ (6 mois)}{Stage professionnel de longue durée -- cursus UTBM}{}{ Parkeon, Besançon} {}{Création d'un logiciel de test pour la gestion de clés cryptographiques de sécurisation de paiements bancaires, développement en Python, C/C++ pour 
l'embarqué}


\cventry{ Juin 2011\\ (6 semaines)}{Stagiaire en première année de BTS IRIS}{}{2AS Informatique} {Besançon}{Maintenance informatique, programmation en Java, conseil clientèle\newline}

\section{Compétences}
\cvitem{Langages}{C, C++, Java, PHP, Bash, VBA, Python, PL/SQL, Ruby}
\cvitem{Framework}{Qt, Django, Flask, wxPython, Ruby on Rails}
\cvitem{Analyse}{UML, Design Patterns, Merise}
\cvitem{Base de données}{MySQL, PostgreSQL, Oracle}
\cvitem{Réseaux}{Certifications Cisco CCNA 1 et CCNA 3}
\cvitem{Logiciels}{Microsoft Office, Libre Office, Adobe Photoshop, iWork, LaTeX}
\cvitem{IDE}{Eclipse(CDT, Pydev), PyCharm, RubyMine}
\cvitem{Anglais}{Lu, écrit, parlé -- Bulats B2 70 / 100}{}
\cvitem{Espagnol}{Lu, écrit, parlé -- Niveau scolaire}{}

\section{Centres d'intérêt}
\cvitem {Informatique}{Intéressé par le logiciel et la culture libre}
\cvitem{Cinéma}{Films et séries TV, Tarantino, Scorsese, J.J Abrams}
\cvitem{Veille tech.}{The Verge, Lifehacker, utilisation de flux RSS }
\cvitem {Lecture}{Comics, mangas, polars}
\cvitem{Jeux vidéo}{Starcraft II, MMORPG}
\cvitem{MOOC}{Cours de l'INRIA	 sur le web sémantique suivis en 2015}
\end{document}