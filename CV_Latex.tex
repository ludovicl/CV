\documentclass[11pt,a4paper]{moderncv}
\moderncvtheme[blue]{classic}              
\usepackage[utf8]{inputenc}
\usepackage{moderntimeline,tikz}

\usepackage[top=1.1cm, bottom=1.1cm, left=1.5cm, right=1.5cm]{geometry}

\setlength{\hintscolumnwidth}{3cm}
\photo[64pt]{Photo.png}

\definecolor{linecolor}{RGB}{128,128,128}
\tlmaxdates{2006}{2015}
%\tlwidth{0.4ex}
%\tltext{\tiny}

\nopagenumbers{}

\firstname{Ludovic}
\familyname{Lardies}
\title{Stage de fin d'études}              
\address{20 rue Octave David}{25000 Besançon}    
\email{ludovic@lardies.fr}                    
\mobile{(033) 06 88 65 64 32} 
\homepage{http://lardies.fr}
\extrainfo{Né le 17/10/1990 \\ Permis B}


\begin{document}

\maketitle
\vspace{-3.5em}


\section{Formations}

\tllabelcventry[linecolor]{2012}{0}{2012}{Ingénieur Informatique (5ème année)}{Université de Technologie de Belfort-Montbéliard UTBM (90)}{}{}{}

\tllabelcventry[linecolor]{2010.5}{2012.5}{2010--2012}{BTS Informatique et Réseaux pour l'Industrie et les Services techniques (IRIS)}{Lycée Jules Haag, Besançon (25)}{}{}{}

\tllabelcventry[linecolor]{2008.5}{2010.5}{2008--2010}{Baccalauréat (Mention AB)}{Lycée Jules Haag, Besançon}{}{}{Série STI, Filiére Microtechniques} 

\tllabelcventry[linecolor]{2006}{2008.5}{2006--2008}{BEP}{Lycée Jules Haag, Besançon}{}{}{Métiers de la production mécanique informatisée} 


\section{Expérience professionnelle}

\cventry{Été 2014\\ (2 mois)}{Interim -- développeur informatique }{}{ Parkeon, Besançon} {}{Création de la documentation d'un SDK. Participation au changement de compilateur et de chaine de compilation (GNU et CMake).}

\cventry{Printemps 2014}{Projet UTBM}{}{Conception des bases de données} {}{Réalisation d'une application de gestion de club de tennis en PL/SQL pour une base de données Oracle.}

\cventry{Automne 2013\\ (6 mois)}{Stage professionnel de longue durée -- cursus UTBM }{}{ Parkeon, Besançon} {}{Création d'un logiciel de test pour la gestion de clés cryptographiques de sécurisation de paiements bancaires, développement en Python, C/C++ pour 
l'embarqué.}

\cventry{Printemps 2013}{Projet UTBM}{}{UV systemes d'information} {}{Réalisation en c (programmation système) d’une simulation de centre d’appel.  \texttt{https://github.com/ludovicl/LO41-Centre-d-appel}}

\cventry{ Juin 2011\\ (6 semaines)}{Stagiaire en première année de BTS IRIS}{}{2AS Informatique} {Besançon}{Maintenance informatique, programmation en Java, conseil clientèle.\newline}

\cventry{Juin 2007 \\ (3 semaines)}{Stagiaire en première année de BEP MPMI}{}{Lip Précision Industrie} {Besançon}{Tournage traditionnel, utilisation de perceuses, tri de matériaux, livraisons.}


\section{Compétences}
\cvitem{Langages}{C, C++, Java, PHP, Bash, VBA, Python, PL/SQL}
\cvitem{Analyse}{UML, Design Patterns, Merise}
\cvitem{Base de données}{MySQL, PostgreSQL, Oracle}
\cvitem{Réseaux}{Certifications Cisco CCNA 1 et CCNA 3}
\cvitem {Logiciels CAO}{Inventor, Solid Works}
\cvitem{Logiciels}{Microsoft office, Open Office, Adobe Photoshop, iWork, LaTeX}
\cvitem{Anglais}{Lu, écrit, parlé -- Bulats B1 69 / 100}{}
\cvitem{Espagnol}{Lu, écrit, parlé -- Niveau scolaire}{}


\section{Centres d'intérêt}
\cvitem {Informatique}{Veille technologique, projets libres}
\cvitem{Cinéma}{Films et séries TV, Tarantino, Scorsese, J.J Abrams}
\cvitem{Sport}{Natation, course à pied}
\cvitem {Lecture}{Comics, mangas, polars}
\cvitem{Jeux video}{Starcraft II, MMORPG}
\end{document}