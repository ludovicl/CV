%%%%%%%%%%%%%%%%%%%%%%%%%%%%%%%%%%%%%%%%%
% Twenty Seconds Resume/CV
% LaTeX Template
% Version 1.0 (14/7/16)
%
% This template has been downloaded from:
% http://www.LaTeXTemplates.com
%
% Original author:
% Carmine Spagnuolo (cspagnuolo@unisa.it) with major modifications by 
% Vel (vel@LaTeXTemplates.com)
%
% License:
% The MIT License (see included LICENSE file)
%
%%%%%%%%%%%%%%%%%%%%%%%%%%%%%%%%%%%%%%%%%

%----------------------------------------------------------------------------------------
%	PACKAGES AND OTHER DOCUMENT CONFIGURATIONS
%----------------------------------------------------------------------------------------

\documentclass[letterpaper]{twentysecondcv} % a4paper for A4

%----------------------------------------------------------------------------------------
%	 PERSONAL INFORMATION
%----------------------------------------------------------------------------------------

% If you don't need one or more of the below, just remove the content leaving the command, e.g. \cvnumberphone{}



\cvname{Ludovic Lardies} % Your name
\profilepic{img/profil.jpg}
\cvjobtitle{Ingénieur informatique, \\ développement logiciel} % Job title/career

\cvlinkedin{https://linkedin.com/in/ludovic-lardies}
\cvnumberphone{06 88 65 64 32} % Phone number
\cvaddress{Paris}  % city
\cvsite{https://lardies.fr} % Personal website
\cvmail{ludovic.lrds@gmail.com} % Email address

%----------------------------------------------------------------------------------------

\begin{document}

\makeprofile % Print the sidebar

\section{Formation}

\begin{twenty} % Environment for a list with descriptions
	\twentyitem
    	{2012 - 2015}
        {École d'ingénieur en informatique}
        {\href{http://www.utbm.fr/}{Université de Technologie de Belfort-Montbéliard, UTBM}}
        {Belfort (90)}
        {Filière Ingénierie des logiciels et de la connaissance}
	\twentyitem
    	{2009 - 2013}
        {BTS Informatique et Réseaux pour l'Industrie et les Services techniques (IRIS)}
        {Lycée Jules Haag}{}
        {Besançon (25)}
	%\twentyitem{<dates>}{<title>}{<organizatidon>}{<location>}{<description>}
\end{twenty}

\section{Expérience professionnelle}
\begin{twenty}
	\twentyitem
    	{Depuis \\ août 2015 \\ (17 mois)}
        {Développeur back-end et ensuite lead développeur back-end}
        {\href{http://www.tech4team.fr}{Tech4Team}}
        {Paris}
        {
        {\begin{itemize}
        	\item Conception et optimisation de bases de données PostgreSQL
	        \item Intégration d’API en Python
	        \item Développement back et front en Ruby on Rails et JQuery
	        \item Administration serveur Debian (Ngnix, FTP, Supervisord, HTTPS)
	        \item AWS : CodeDeploy, EC2, VPC, RDS
	    \end{itemize}}
        }
    \twentyitem
    	{Fév. 2015 - \\ Juil. 2015 \\ (6 mois)}
        {Stage de fin d'études - cursus UTBM}
        {\href{http://www.tech4team.fr}{Tech4Team}}
        {Paris}
        {
        {\begin{itemize}
        	\item Parsing et importation de données clients (fichiers CSV ou utilisation d'API) dans une base de données PostgreSQL (en Python et Ruby)
	        \item Scrapping de sites en Python
	    \end{itemize}}
        }
     \twentyitem
    	{Sept. 2013 - \\ Fév. 2014 \\ (6 mois)}
        {Stage professionnel de longue dureée – cursus UTBM}
        {\href{http://parkeon.fr}{Parkeon}}
        {Besançon (25)}
        {
        {\begin{itemize}
        	\item Création d'un logiciel de test pour la gestion de clés cryptographiques de sécurisation de paiements bancaires en Python
	        \item Développement en C/C++ pour l’embarqué
	    \end{itemize}}
        }
\end{twenty}

\section{MOOC}

\begin{twenty}
    
    \twentyitem
    	{Début 2016}
        {MOOC de Université Paris-Sud}
        {\href{http://www.fun-mooc.fr/courses/UPSUD/42001S06/session06/about}{FUN-MOOC}}
        {}
        {
        {Introduction à la statistique avec R (note de 73\%)}
        }
        
    \twentyitem
    	{Début 2015}
        {MOOC de l’INRIA}
        {\href{https://www.fun-mooc.fr}{FUN-MOOC}}
        {}
        {
        {Web sémantique et Web de données (note de 85\%)}
        }

\end{twenty}

\section{Centres d'intérêt}

\begin{twenty}    
    \twentyitem
    	{Informatique}
    	{}
    	{}
        {Intéressé par le logiciel et la culture libre}
        {}
	
	  \twentyitem
    	{Cinéma}
    	{}
    	{}
        {Films et séries TV, Tarantino, Scorsese, J.J Abrams}
        {}
        
     \twentyitem
    	{Veille tech.}
    	{}
    	{}
        {The Verge, Lifehacker, utilisation de flux RSS}
        {}


        
       
        

\end{twenty}


\end{document} 
